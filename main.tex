\documentclass[conference]{IEEEtran}
\IEEEoverridecommandlockouts
% The preceding line is only needed to identify funding in the first footnote. If that is unneeded, please comment it out.
\usepackage{cite}
\usepackage{amsmath,amssymb,amsfonts}
\usepackage{algorithmic}
\usepackage{graphicx}
\usepackage{textcomp}
\usepackage{xcolor}
\usepackage{appendix}
\usepackage{amsmath}
 \usepackage{multirow}
\usepackage[margin=1in]{geometry}
\usepackage{amsmath,amssymb,amsfonts}
\usepackage{algorithmic}
\usepackage{graphicx}
\usepackage{textcomp}
\usepackage{xcolor}
\def\BibTeX{{\rm B\kern-.05em{\sc i\kern-.025em b}\kern-.08em
    T\kern-.1667em\lower.7ex\hbox{E}\kern-.125emX}}
% \usepackage{booktabs, multirow} % for borders and merged ranges
% \usepackage{soul}% for underlines
% \usepackage[table]{xcolor} % for cell colors
% \usepackage{changepage,threeparttable} % for wide tables

\makeatletter
\newcommand{\linebreakand}{%
  \end{@IEEEauthorhalign}
  \hfill\mbox{}\par
  \mbox{}\hfill\begin{@IEEEauthorhalign}
}
\makeatother

\begin{document}

\title{Stats 230 - Final Project\\}

\author{\IEEEauthorblockN{Andrew Cambridge}
% \IEEEauthorblockA{\textit{dept. name of organization (of Aff.)} \\
% \textit{name of organization (of Aff.)}\\
% City, Country \\
% email address}
\and
\IEEEauthorblockN{Kirsten Chapman}
% \IEEEauthorblockA{\textit{dept. name of organization (of Aff.)} \\
% \textit{name of organization (of Aff.)}\\
% City, Country \\
% email address}
\and
\IEEEauthorblockN{Madison Wozniak}
% \IEEEauthorblockA{\textit{dept. name of organization (of Aff.)} \\
% \textit{name of organization (of Aff.)}\\
% City, Country \\
% email address}
\linebreakand 
\IEEEauthorblockN{Sterling Bradshaw}
% \IEEEauthorblockA{\textit{dept. name of organization (of Aff.)} \\
% \textit{name of organization (of Aff.)}\\
% City, Country \\
% email address}
}
\maketitle

\begin{abstract}
Abstract? 
\end{abstract}

\begin{IEEEkeywords}
Operating System, Android, iPhone, Human-Computer Interaction
\end{IEEEkeywords}

\section{Introduction}

The debate between which smartphone operating system is "best" can be quite contentious: with some phone users staunchly defending iOS and others fully supporting Android. As such, we seek out to quantify which operating system is truly optimal. For this study, we will be examining the ease of use that users have with both operating systems; looking specifically at the time it takes for them to complete phone-based tasks. Using this time data, we seek to determine which operating system is truly more 'user friendly' by examining the differences in speed between the two operating systems taking into account the participants' native operating system (i.e., the phone they currently use). Consequently, our research question is: \\
\textbf{(RQ1a):} Is Android or iOS the more user friendly operating system based on speed? \\
\textbf{(RQ1b):} How does a users' native operating system effect this? \\
As we begin our study, we expect that the effect of operating system on the time it takes to complete a task will change based on whether or not it is the user's native operating system. 

\begin{table*}[ht]
\centering
\begin{tabular}{ll|lllll}
\textbf{Error}                       &                                  & \textbf{df} & \textbf{SS} & \textbf{MS} & \textbf{F-Stat} & \textbf{P-Val} \\ \hline
\multirow{2}{*}{\textbf{Subject ID}} & \textbf{Native Brand}            & 1           & 37463       & 37463       & 3.347           & 0.0767         \\
                                     & \textbf{Residuals}               & 32          & 358199      & 11194       &                 &                \\ \hline
\multirow{7}{*}{\textbf{Within}}     & \textbf{Phone Type}              & 1           & 11711       & 11711       & 2.262           & 0.13591        \\
                                     & \textbf{Task Type}               & 1           & 173816      & 173816      & 33.567          & 8.72e-08 ***   \\
                                     & \textbf{Native Brand:Phone Type} & 1           & 87459       & 87459       & 16.890          & 8.35e-05 ***   \\
                                     & \textbf{Native Brand:Task Type}  & 1           & 55408       & 55408       & 10.700          & 0.00149 **     \\
                                     & \textbf{Phone Type:Task Type}    & 1           & 51792       & 51792       & 10.002          & 0.00209 **     \\
 & \textbf{\begin{tabular}[c]{@{}l@{}}Native Brand:Phone Type:\\ Task Type\end{tabular}} & 1 & 62093 & 62093 & 11.991 & 0.00080 *** \\
                                     & \textbf{Residuals}               & 96          & 497109      & 5178        &                 &               
\end{tabular}
\caption{p \textless 0.05: *; p \textless 0.01: **; p \textless 0.001: ***}
\label{tab:anova}
\end{table*}

\section{Design and Data Collection} 
\subsection{Experimental Design}
The experimental design we used was a split-plot/repeated measures design with one between-unit factor and two within-unit factors (SP/RM[1;2]). Our large experimental unit was a person, and the between-units factor assigned to person was native phone type of the subject (iPhone, Android, or Other). The experimental unit within the person block is a time within a person, which was assigned a task type and a phone type. Our tasks had two levels, task 1 and task 2. Phone type also included two levels, iPhone and Android. Figure 1 shows the diagram of our experimental design. 
\begin{figure}
    \centering
    \includegraphics[width = 7cm]{native type copy.jpeg}
    \caption{The experimental design of our split plot/repeated measures study}
    \label{fig:fig1}
\end{figure}

\subsection{Statistical Model}
The model we used for the SP/RM[1;2] experiment was $Y_{ijkl} = \mu + \alpha_{i} + \beta_{j(i)} + \gamma_{k} + \eta_{l} + (\alpha\gamma)_{ik} + (\alpha\eta)_{il} + (\gamma\eta)_{kl} + (\alpha\gamma\eta)_{ikl} + \epsilon_{ijkl}$ with $Y_{ijkl}$ representing each observation, $\mu$ representing the grand mean, $\alpha_{i}$ representing the native phone type effect with two levels (iPhone, Android), $\beta_{j(i)}$ representing each person, $\gamma_{k}$ representing the phone type effect with two levels (iPhone, Android), $\eta_{l}$ representing  the task type effect with two levels (screenshot task, settings task), $(\alpha\gamma)_{ik}$ representing the native phone type-phone type interaction effect, $(\alpha\eta)_{il}$ representing the native phone type-task type interaction effect, $(\gamma\eta)_{kl}$ representing the phone type-task type interaction, $(\alpha\gamma\eta)_{ikl}$ representing the 3-way interaction effect of all three factors, and $\epsilon_{ijkl}$ representing the residual.
The hypotheses we want to test are the following: \\
1. $H_0: \alpha_1 = \alpha_2$ vs. $H_A: \alpha_1 \neq \alpha_2$ \\
2. $H_0: \gamma_1 = \gamma_2$ vs. $H_A: \gamma_1 \neq \gamma_2$\\
3. $H_0: \eta_1 = \eta_2$ vs. $H_A: \eta_1 \neq \eta_2$\\
4. $H_0: (\alpha\gamma)_{11} = \cdots = (\alpha\gamma)_{22} = 0$ vs. $H_A:$ At least one of the $(\alpha\gamma)_{ik}$ is different from the others \\
5. $H_0: (\alpha\eta)_{11} = \cdots = (\alpha\eta)_{22} = 0$ vs. $H_A:$ At least one of the $(\alpha\eta)_{il}$ is different from the others \\
6. $H_0: (\gamma\eta)_{11} = \cdots = (\gamma\eta)_{22} = 0$ vs. $H_A:$ At least one of the $(\gamma\eta)_{kl}$ is different from the others \\
7. $H_0: (\alpha\gamma\eta)_{111} = \cdots = (\alpha\gamma\eta)_{222} = 0$ vs. $H_A:$ At least one of the $(\alpha\gamma\eta)_{ikl}$ is different from the others \\

\subsection{Power Analysis}
Based on the 2 groups of between block factors (native\_type with levels of Android and iOS), our 34 participants enable us to detect a difference of 30 seconds between software types, given expected standard deviation of 33 (anything beyond 100 seconds from the mean would be considered too weird) and a significance level of .05, with approximately 95\% power. A significant issue with this experiment is that the actual standard error of the data was about 99.44. While our expected standard deviation indicates 95\% power, power calculated with the standard error of the data is only 23\%, not at all large enough to make any claims with certainty. The remainder of this report discusses significance of factor effects based on p-values, but be aware that the conclusions are grossly under powered.

\subsection{Data Gathering Methodology}
Data was gathered using the procedure outlined in our script (see Appendix). Each participant was given instructions about how the experiment would go. They were told not to use the search function, when to start, and what to say when they were done. Before each observation, the participant was allowed to review the task they were to perform. The person was then given a phone and a timer started until they completed the task (see the tasks listed in the script found in the Appendix).

Each of the tasks was done on each of the two phones. Both the order in which the phones were given and the order of the tasks within each phone were randomly assigned. The following code was used to create the random order for each participant, followed by how each number was interpreted. The randomization of how the phones and tasks were intermixed allowed us to randomize potentially confounding effects caused by the order of the phones or the order of the tasks, and thus manage and deal with variation caused by things such as learning curves and similarities or differences between tasks and phones. If a participant could not complete the task, their time was recorded as the maximum time of 10 minutes (600 seconds).


\section{Data Analysis}

Refer to table ~\ref{tab:anova} for our Anova table. 

\subsection{ANOVA Assumptions}
The normality assumption was checked both within subjects and between subjects. Graphs of each of their histograms indicate both error distributions resemble normal distributions. To check that the errors have common standard deviations, the standard deviations of both errors were calculated, with the resulting values: between subjects had a standard deviation of 103.485 and within-subject was 70.0545. Because the ratio of standard deviations is less than 2:1, they have similar standard deviations (code for these graphs and calculations can be found in the Appendix). 

\subsection{Factor Effects}
With a .05 significance level, we see that all our within block factors other than the phone type were statistically significant. The Native Brand, a between block factor, was not significant. Thus, we fail to reject the null hypotheses that effect sizes are different from each other and zero for both the Native Brand main effect and Phone Type main effect. We reject the null hypotheses for all other main effects and interaction effects in the model. 

We will examine exactly how much of the variability in the response variable was due to each factor individually using each factor's $\eta^2$.\\
$\eta^2_{task} \approx 0.19$ tells us that the task type factor accounts for about $19\%$ of the variability in the response, $\eta^2_{native*phone} \approx 0.093$ tells us that the native type by phone type interaction factor accounts for about $9\%$ of the variability in the response, $\eta^2_{native*task} \approx 0.059$ tells us that the native type by task type factor accounts for about $6\%$ of the variability in the response, $\eta^2_{phone*task} \approx 0.055$ tells us that the phone type by task type accounts for about $6\%$ of the variability in the response, and $\eta^2_{native*phone*task} \approx 0.066$. tells us that the 3-way interaction accounts for about $7\%$ of the variability in the response.\\
In response to our first research question, we conclude that with a p-value of $0.136$ the phone type does not affect the time it takes for a subject to complete a task on the phone. Our other question was if the subjects native phone type would affect the response time for the different levels of phone type (or in other words, if there is a significant interaction between the native phone type and phone type factors). With a p-value of 0.000084, this interaction effect is definitely significant. From the $\eta^2_{native*phone}$ we see this interaction accounts for about $9\%$ of the variability in the response variable. Figure 2 shows our interaction graph. 

\begin{figure}
    \centering
    \includegraphics[width = 7cm]{stat230_interaction.png}
    \caption{The interaction between Native\_Brand and Phone\_Type}
    \label{fig:fig2}
\end{figure}

\section{Conclusion}
% \textcolor{red}{Briefly summarize important findings: \\ 1. The phone type was NOT significant \\ 2. The interaction of phone type and native type \\ 3. Our task types were hugely significant \\4. Based on our findings, a BYU student that uses an Android may find it harder to move to iPhone than an iPhone user would find it to move to Android. Apple users average the same times between Apple and Android} \\
% \textcolor{red}{Discuss ways in which your findings can be interpreted and generalized.  Is there a cause-and-effect relationship (or, if the factor was not significant, COULD it have been considered to be a causal effect)?  What populations can you reasonably draw inferences about? \\  1. This was not representative of entire population.. primarily looked at students. We made random assignments so we can state some cause-and-effect but we can't make a solid inference about the rest of the population because these were not random selections. \\ 2.  Based off what we saw in our sample, the difference in our operating systems is less important than native brand}
% \textcolor{red}{How might a future study be refined and/or expanded with extra time or resources? \\ 1. Representative sampling instead of convenience sampling. \\ 2. Increasing sample size would increase power in detecting differences. If we had a bigger sample size we could detect a smaller difference (currently 30 seconds). }

Three key findings emerged from our data and analysis. Firstly, phone type alone was not a significant factor in the time it took a participant to complete the tasks. This indicates that with all else held constant, switching the phone that a participant was using for a certain task would have no significant effect on the time it took them to complete said task. Secondly, we found that the task types themselves were highly significant. Meaning that a change in task had a significant effect on the time it took to complete a task. Finally, we found that the interaction of phone type and native type was significant. This possibly indicates that a BYU student who uses an Android phone may find it harder to move to an iPhone than an iPhone user would find it to move to an Android. 

These conclusions are supported by the p-values created in the anova table

Our study sample was not representative of the entire population, it primarily consisted of BYU students. As we used convenience sampling, we cannot make any solid inferences about the broader population. We can though state some cause-and-effect since we randomly assigned the order in that participants would complete the task.  

In order to mitigate some of the limitations of our work, future studies should gather a larger, demographically representative sample. Increasing the sample size will also increase the power, and in turn, will allow us to detect smaller differences in the data (we can currently only detect a 30-second difference). This experiment could have been improved by tighter quality control of the treatment process and environment. Due to a larger standard deviation than expected, this experiment was also significantly under powered and would be able to yield reliable results given the increased sample sized mentioned above.

\onecolumn
\appendix
\paragraph{Anova Code}
\hfill\break \noindent library(tidyverse)\\
\textcolor{gray}{\# the below code sets the factors in the data} \\
\noindent phones\$Native\_Brand $\leftarrow$ as\_factor(phones\$Native\_Brand)\\
phones\$Phone\_Type $\leftarrow$ as\_factor(phones\$Phone\_Type)\\
phones\$Task\_Type $\leftarrow$ as\_factor(phones\$Task\_Type)\\
phones\$Subject\_ID $\leftarrow$ as\_factor(phones\$Subject\_ID)\\

\noindent \textcolor{gray}{\#performs the anova analysis on our statistical model and provides the anova table summary}\\
\noindent phones.aov $\leftarrow$ aov(Completion\_Time $\sim$ Native\_Brand + Error(Subject\_ID) + Phone\_Type + Task\_Type + Native\_Brand:Phone\_Type 
                  + Native\_Brand:Task\_Type + Phone\_Type:Task\_Type + Native\_Brand:Phone\_Type:Task\_Type, data=phones)\\ 
summary(phones.aov)\\


\paragraph{Power Test Code}


\hfill\break \noindent library(tidyverse)\\
\textcolor{gray}{\#determines what level of power we will have with the given information} \\
power.anova.test(groups=2, between.var=var(c(0,30)), within.var=$(100/3)^2$, sig.level=.05, n=34)\\
phones $\leftarrow$ read\_csv("Phones.csv")\\
sd(phones\$Completion\_Time)\\

\paragraph{Order Randomization Code}

\hfill\break \noindent sample(1:4,4)\\
\textcolor{gray}{Where: \\
1 = iPhone, Task 1\\
2 = iPhone, Task 2\\
3 = Android, Task 1\\
4 = Android, Task 2 }\\

\paragraph{Histograms} 
\hfill\break \noindent \textcolor{gray}{\# graphing histograms of both between and within residuals to check normality assumption} \\
hist(phones.aov\$Within\$residuals) \\
hist(phones.aov\$Subject\_ID\$residuals) \\ \\ 
\begin{figure}[htbp]
    \centering
    \includegraphics[width = 10cm]{histograms.png}
    \caption{}
    \label{fig:fig3}
\end{figure} 
\noindent 


\paragraph{S Assumptions Code}

\hfill\break\noindent \textcolor{gray}{\# compute standard deviations of within and between residuals for similar standard deviation assumptions} \\
\noindent sd(phones.aov\$Subject\_ID\$residuals) \\
\noindent sd(phones.aov\$Within\$residuals) \\ 




\paragraph{Protocol}
\hfill\break \noindent \textbf{Introductory Script:} \\
Determine the native type of the participant: On a day-to-day basis, what type of phone do you use? iPhone, Android, other.
You will be given two working and unlocked phones and asked to complete two tasks on each of them. I will randomly give you one of the phones and then give you instructions on how to complete a randomly determined task. We will do this four times, such that you will complete each task on each phone. You may not use the search function to look for apps or settings. Please hold the phone screen so that I can see it and say “done” when you have completed the task. 
Here are the instructions for the first task. Please read them and then we will start in 30 seconds. Give the person the instruction sheet for a random task and start a 30 second timer. 
At the end of 30 seconds: Start the task when I say “go.” Place the unlocked phone in front of the person and get your timer ready. “Go.” Start the timer when you say go. 
Stop the timer when the person has said “done” and you can see that the task is complete. Follow along the entire time, without interrupting. If the person simply cannot complete the task or gives up, mark the time as “INF.”  \\ 
\textbf{Task 1} \\
Take a screenshot of the home screen.
Save the screenshot to wherever the phone stores photos.
Locate the image in the application used by the phone to store photos.
Edit the image by applying a preset filter (does not have to be a specific filter, just any).
Change the caption or title of the image to “you\_are\_awesome123!” \\
\textbf{Task 2} \\ 
Using the phone’s settings application to change the time to 24-hour/military time format.
Also in settings, turn the device into night/dark mode.
Lastly, find the area of settings to manage battery health and turn the device onto power saving mode.

\paragraph{Data Sheet}
%Please add the following packages if necessary:
%\usepackage{booktabs, multirow} % for borders and merged ranges
%\usepackage{soul}% for underlines
%\usepackage[table]{xcolor} % for cell colors
%\usepackage{changepage,threeparttable} % for wide tables
%If the table is too wide, replace \begin{table}[!htp]...\end{table} with
%\begin{adjustwidth}{-2.5 cm}{-2.5 cm}\centering\begin{threeparttable}[!htb]...\end{threeparttable}\end{adjustwidth}
\begin{table}[!htp]\centering
\caption{Generated by Spread-LaTeX}\label{tab: }
\scriptsize
\begin{tabular}{lrrrrrrrr}\toprule
Observation\_Number &Subject\_ID &Proctor &Native\_Brand &Phone\_Type &Type\_Match &Task\_Type &Completion\_Time \\\midrule
1 &1 &Andrew &Android &iPhone &false &Task 1 - screenshot &238 \\
2 &1 &Andrew &Android &Android &true &Task 2 - settings &71 \\
3 &1 &Andrew &Android &Android &true &Task 1 - screenshot &80 \\
4 &1 &Andrew &Android &iPhone &false &Task 2 - settings &108 \\
5 &2 &Andrew &Apple/iOS &Android &false &Task 1 - screenshot &300 \\
6 &2 &Andrew &Apple/iOS &iPhone &true &Task 2 - settings &246 \\
7 &2 &Andrew &Apple/iOS &iPhone &true &Task 1 - screenshot &162 \\
8 &2 &Andrew &Apple/iOS &Android &false &Task 2 - settings &98 \\
9 &3 &Andrew &Apple/iOS &iPhone &true &Task 1 - screenshot &93 \\
10 &3 &Andrew &Apple/iOS &Android &false &Task 1 - screenshot &110 \\
11 &3 &Andrew &Apple/iOS &Android &false &Task 2 - settings &93 \\
12 &3 &Andrew &Apple/iOS &iPhone &true &Task 2 - settings &27 \\
13 &4 &Andrew &Apple/iOS &iPhone &true &Task 2 - settings &41 \\
14 &4 &Andrew &Apple/iOS &Android &false &Task 1 - screenshot &173 \\
15 &4 &Andrew &Apple/iOS &Android &false &Task 2 - settings &114 \\
16 &4 &Andrew &Apple/iOS &iPhone &true &Task 1 - screenshot &120 \\
17 &5 &Kirsten &Apple/iOS &Android &false &Task 2 - settings &40 \\
18 &5 &Kirsten &Apple/iOS &iPhone &true &Task 2 - settings &72 \\
19 &5 &Kirsten &Apple/iOS &iPhone &true &Task 1 - screenshot &180 \\
20 &5 &Kirsten &Apple/iOS &Android &false &Task 1 - screenshot &135 \\
21 &6 &Kirsten &Apple/iOS &iPhone &true &Task 1 - screenshot &72 \\
22 &6 &Kirsten &Apple/iOS &Android &false &Task 2 - settings &85 \\
23 &6 &Kirsten &Apple/iOS &iPhone &true &Task 2 - settings &51 \\
24 &6 &Kirsten &Apple/iOS &Android &false &Task 1 - screenshot &120 \\
25 &7 &Kirsten &Apple/iOS &Android &false &Task 2 - settings &81 \\
26 &7 &Kirsten &Apple/iOS &iPhone &true &Task 1 - screenshot &114 \\
27 &7 &Kirsten &Apple/iOS &Android &false &Task 1 - screenshot &72 \\
28 &7 &Kirsten &Apple/iOS &iPhone &true &Task 2 - settings &48 \\
29 &8 &Kirsten &Apple/iOS &Android &false &Task 1 - screenshot &124 \\
30 &8 &Kirsten &Apple/iOS &iPhone &true &Task 2 - settings &78 \\
31 &8 &Kirsten &Apple/iOS &iPhone &true &Task 1 - screenshot &39 \\
32 &8 &Kirsten &Apple/iOS &Android &false &Task 2 - settings &182 \\
33 &9 &Kirsten &Apple/iOS &Android &false &Task 1 - screenshot &96 \\
34 &9 &Kirsten &Apple/iOS &iPhone &true &Task 2 - settings &80 \\
35 &9 &Kirsten &Apple/iOS &iPhone &true &Task 1 - screenshot &88 \\
36 &9 &Kirsten &Apple/iOS &Android &false &Task 2 - settings &106 \\
37 &10 &Kirsten &Apple/iOS &iPhone &true &Task 2 - settings &94 \\
38 &10 &Kirsten &Apple/iOS &iPhone &true &Task 1 - screenshot &64 \\
39 &10 &Kirsten &Apple/iOS &Android &false &Task 1 - screenshot &124 \\
40 &10 &Kirsten &Apple/iOS &Android &false &Task 2 - settings &137 \\
41 &11 &Sterling &Apple/iOS &iPhone &true &Task 2 - settings &53 \\
42 &11 &Sterling &Apple/iOS &Android &false &Task 2 - settings &62 \\
43 &11 &Sterling &Apple/iOS &Android &false &Task 1 - screenshot &170 \\
44 &11 &Sterling &Apple/iOS &iPhone &true &Task 1 - screenshot &200 \\
45 &12 &Sterling &Android &iPhone &false &Task 2 - settings &60 \\
46 &12 &Sterling &Android &Android &true &Task 2 - settings &72 \\
47 &12 &Sterling &Android &Android &true &Task 1 - screenshot &179 \\
48 &12 &Sterling &Android &iPhone &false &Task 1 - screenshot &600 \\
49 &13 &Sterling &Apple/iOS &iPhone &true &Task 2 - settings &36 \\
50 &13 &Sterling &Apple/iOS &Android &false &Task 2 - settings &91 \\
51 &13 &Sterling &Apple/iOS &iPhone &true &Task 1 - screenshot &61 \\
52 &13 &Sterling &Apple/iOS &Android &false &Task 1 - screenshot &82 \\
53 &14 &Sterling &Apple/iOS &iPhone &true &Task 2 - settings &53 \\
54 &14 &Sterling &Apple/iOS &Android &false &Task 1 - screenshot &172 \\
55 &14 &Sterling &Apple/iOS &iPhone &true &Task 1 - screenshot &98 \\
56 &14 &Sterling &Apple/iOS &Android &false &Task 2 - settings &135 \\
57 &15 &Sterling &Android &iPhone &false &Task 2 - settings &55 \\
58 &15 &Sterling &Android &Android &true &Task 2 - settings &73 \\
59 &15 &Sterling &Android &Android &true &Task 1 - screenshot &74 \\
60 &15 &Sterling &Android &iPhone &false &Task 1 - screenshot &120 \\
61 &16 &Sterling &Apple/iOS &Android &false &Task 1 - screenshot &62 \\
62 &16 &Sterling &Apple/iOS &Android &false &Task 2 - settings &42 \\
63 &16 &Sterling &Apple/iOS &iPhone &true &Task 2 - settings &18 \\
64 &16 &Sterling &Apple/iOS &iPhone &true &Task 1 - screenshot &137 \\
65 &17 &Sterling &Android &Android &true &Task 2 - settings &80 \\
66 &17 &Sterling &Android &Android &true &Task 1 - screenshot &93 \\
67 &17 &Sterling &Android &iPhone &false &Task 1 - screenshot &135 \\
68 &17 &Sterling &Android &iPhone &false &Task 2 - settings &42 \\
69 &18 &Sterling &Android &Android &true &Task 1 - screenshot &184 \\
70 &18 &Sterling &Android &Android &true &Task 2 - settings &70 \\
71 &18 &Sterling &Android &iPhone &false &Task 1 - screenshot &600 \\
72 &18 &Sterling &Android &iPhone &false &Task 2 - settings &79 \\
73 &19 &Sterling &Apple/iOS &iPhone &true &Task 2 - settings &44 \\
74 &19 &Sterling &Apple/iOS &Android &false &Task 1 - screenshot &184 \\
75 &19 &Sterling &Apple/iOS &iPhone &true &Task 1 - screenshot &69 \\
76 &19 &Sterling &Apple/iOS &Android &false &Task 2 - settings &60 \\
77 &20 &Sterling &Apple/iOS &Android &false &Task 1 - screenshot &78 \\
78 &20 &Sterling &Apple/iOS &Android &false &Task 2 - settings &52 \\
79 &20 &Sterling &Apple/iOS &iPhone &true &Task 2 - settings &36 \\
80 &20 &Sterling &Apple/iOS &iPhone &true &Task 1 - screenshot &184 \\
\bottomrule
\end{tabular}
\end{table}

\begin{table}[!htp]\centering
\caption{Generated by Spread-LaTeX}\label{tab: }
\scriptsize
\begin{tabular}{lrrrrrrrr}\toprule
Observation\_Number &Subject\_ID &Proctor &Native\_Brand &Phone\_Type &Type\_Match &Task\_Type &Completion\_Time \\\midrule
81 &21 &Madison &Android &Android &true &Task 2 - settings &79 \\
82 &21 &Madison &Android &iPhone &false &Task 2 - settings &47 \\
83 &21 &Madison &Android &iPhone &false &Task 1 - screenshot &124 \\
84 &21 &Madison &Android &Android &true &Task 1 - screenshot &59 \\
85 &22 &Madison &Apple/iOS &iPhone &true &Task 2 - settings &47 \\
86 &22 &Madison &Apple/iOS &Android &false &Task 1 - screenshot &61 \\
87 &22 &Madison &Apple/iOS &Android &false &Task 2 - settings &58 \\
88 &22 &Madison &Apple/iOS &iPhone &true &Task 1 - screenshot &53 \\
89 &23 &Madison &Apple/iOS &Android &false &Task 1 - screenshot &91 \\
90 &23 &Madison &Apple/iOS &Android &false &Task 2 - settings &70 \\
91 &23 &Madison &Apple/iOS &iPhone &true &Task 1 - screenshot &49 \\
92 &23 &Madison &Apple/iOS &iPhone &true &Task 2 - settings &31 \\
93 &24 &Madison &Android &iPhone &false &Task 1 - screenshot &78 \\
94 &24 &Madison &Android &Android &true &Task 1 - screenshot &55 \\
95 &24 &Madison &Android &iPhone &false &Task 2 - settings &47 \\
96 &24 &Madison &Android &Android &true &Task 2 - settings &39 \\
97 &25 &Madison &Android &Android &true &Task 2 - settings &205 \\
98 &25 &Madison &Android &iPhone &false &Task 2 - settings &133 \\
99 &25 &Madison &Android &Android &true &Task 1 - screenshot &158 \\
100 &25 &Madison &Android &iPhone &false &Task 1 - screenshot &600 \\
101 &26 &Madison &Apple/iOS &iPhone &true &Task 1 - screenshot &118 \\
102 &26 &Madison &Apple/iOS &Android &false &Task 2 - settings &138 \\
103 &26 &Madison &Apple/iOS &Android &false &Task 1 - screenshot &101 \\
104 &26 &Madison &Apple/iOS &iPhone &true &Task 2 - settings &59 \\
105 &27 &Madison &Android &iPhone &false &Task 1 - screenshot &150 \\
106 &27 &Madison &Android &iPhone &false &Task 2 - settings &72 \\
107 &27 &Madison &Android &Android &true &Task 1 - screenshot &77 \\
108 &27 &Madison &Android &Android &true &Task 2 - settings &92 \\
109 &28 &Madison &Android &Android &true &Task 1 - screenshot &104 \\
110 &28 &Madison &Android &iPhone &false &Task 1 - screenshot &231 \\
111 &28 &Madison &Android &Android &true &Task 2 - settings &71 \\
112 &28 &Madison &Android &iPhone &false &Task 2 - settings &37 \\
113 &29 &Madison &Android &Android &true &Task 2 - settings &20 \\
114 &29 &Madison &Android &iPhone &false &Task 1 - screenshot &600 \\
115 &29 &Madison &Android &iPhone &false &Task 2 - settings &78 \\
116 &29 &Madison &Android &Android &true &Task 1 - screenshot &139 \\
117 &30 &Andrew &Apple/iOS &iPhone &true &Task 2 - settings &69 \\
118 &30 &Andrew &Apple/iOS &Android &false &Task 1 - screenshot &110 \\
119 &30 &Andrew &Apple/iOS &iPhone &true &Task 1 - screenshot &71 \\
120 &30 &Andrew &Apple/iOS &Android &false &Task 2 - settings &79 \\
121 &31 &Andrew &Android &iPhone &false &Task 2 - settings &59 \\
122 &31 &Andrew &Android &iPhone &false &Task 1 - screenshot &178 \\
123 &31 &Andrew &Android &Android &true &Task 1 - screenshot &86 \\
124 &31 &Andrew &Android &Android &true &Task 2 - settings &73 \\
125 &32 &Andrew &Android &Android &true &Task 1 - screenshot &84 \\
126 &32 &Andrew &Android &iPhone &false &Task 1 - screenshot &77 \\
127 &32 &Andrew &Android &Android &true &Task 2 - settings &45 \\
128 &32 &Andrew &Android &iPhone &false &Task 2 - settings &25 \\
129 &33 &Kirsten &Apple/iOS &iPhone &true &Task 2 - settings &72 \\
130 &33 &Kirsten &Apple/iOS &iPhone &true &Task 1 - screenshot &103 \\
131 &33 &Kirsten &Apple/iOS &Android &false &Task 1 - screenshot &122 \\
132 &33 &Kirsten &Apple/iOS &Android &false &Task 2 - settings &75 \\
133 &34 &Kirsten &Android &Android &true &Task 2 - settings &53 \\
134 &34 &Kirsten &Android &iPhone &false &Task 1 - screenshot &97 \\
135 &34 &Kirsten &Android &iPhone &false &Task 2 - settings &48 \\
136 &34 &Kirsten &Android &Android &true &Task 1 - screenshot &86 \\
\bottomrule
\end{tabular}
\end{table}
\end{document}
